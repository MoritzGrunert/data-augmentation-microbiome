\chapter{Background}

~\cite{sayyari2014}
\begin{itemize}
\item Die Trainingsdaten für die Mikrobiomklassifizierung sind eine OTU-Tabelle (Operational Taxonomic Unit) X.
    \begin{itemize}
        \item Zeile = Inidvidium (Probe)
        \item Spalte = OTU (Taxonomische Einheit)
        \item Wert = Abundanz der OTU in der Probe
        \item Können normalisiert werden (Zeilenweise auf 1 summiert)
        \item label y für jede Probe (z.B. gesund/krank, schlank/übergewichtig)
        \item Diese Sequenzen können mit verschiedenen Ansätzen gewonnen werden, darunter die traditionellen OTU-Auswahlmethoden (Edgar, 2010; Schloss und Handelsman, 2005) oder Methoden mit sub-operationalen taxonomischen Einheiten (Amir et al., 2017; Callahan et al., 2016; Edgar, 2016)
    \end{itemize}
\end{itemize}