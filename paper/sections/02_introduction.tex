\chapter{Introduction}

\section{Motivation}
In recent years, machine learning (ML) methods have demonstrated an advanced capacity to capture complex relationships. Nevertheless, the complex nature of microbiome data makes it challenging for conventional machine learning (ML) classification models to achieve suitable performance~\cite{sayyari2014}.
The microbiome is defined as a collective of microorganisms that inhabit a specific body site or environment~\cite{sharma2024,ursell2012}.
In addition, the number of training samples for the machine learning (ML) model can have a substantial impact on the model's accuracy. A limited dataset can result in overfitting and suboptimal generalization~\cite{sayyari2014,sharma2024}.
This issue is particularly salient in the context of microbiome datasets, which their are often limited in size, this can result in a higher number of features compared to the number of samples. Additionally, these datasets are frequently imbalanced and biased, a circumstance that can also compromise the accuracy of the model~\cite{sayyari2014,sharma2024}.

A potential solution to avoid a suboptimal model caused by an an unfavorable dataset, could be the implementation of data augmentation techniques. The concept of data augmentation involves the generation of synthetic samples based on real samples and adding these new samples to the training data~\cite{sayyari2014}.
Two widely used data augmentation methods for handling class imbalance are SMOTE~\cite{chawla2002} and ADASYN~\cite{he2008}. Both approaches aim to mitigate biases introduced by imbalanced class distributions by generating synthetic samples for the minority class. This is achieved employing a k-nearest neighbors (k-NN) strategy that interpolates between similar minority-class samples. However, these two methods do not attempt to capture domain knowledge, such as the relationship between microorganisms~\cite{sayyari2014}.
In this paper, we examine more closely two Data Augmentaion techniques. The Tree-based Associative Data Augmentation (TADA) approach was proposed by Sayyari et al. in 2014. TADA aims to generate new samples by considering the evolutionary relationships between organisms using a phylogenetic tree. A phylogenetic tree is a representation of evolutionary relationships among species and the branches representing the evolutionary distance between them~\cite{sayyari2014}.
The objective of this study is to empirically evaluate the efficacy of various data augmentation approaches, including SMOTE, ADASYN, TADA, and DBS, on a single dataset characterized by distinct training data distributions. The study utilized two distinct classification algorithms: random forest and logistic regression. The findings demonstrate that all augmentation methods enhance model performance, but TADA and DBS show more efficacy.

\section{Thesis Outline}
Your outline text here.

